% This is samplepaper.tex, a sample chapter demonstrating the
% LLNCS macro package for Springer Computer Science proceedings;
% Version 2.20 of 2017/10/04
%
\documentclass[runningheads]{llncs}
\renewcommand{\baselinestretch}{1.1} 
%
\usepackage{graphicx}
\usepackage{xcolor}
\usepackage{xspace}
\usepackage{hyperref}
\usepackage{subcaption}
\usepackage{listings}
\usepackage{ulem}
% \usepackage{geometry}
% \usepackage{marginnote}

% Fix link colors
\hypersetup{
    colorlinks = true,
    linkcolor=red,
    citecolor=red,
    urlcolor=blue,
    linktocpage % so that page numbers are clickable in toc
}

% Used for displaying a sample figure. If possible, figure files should
% be included in EPS format.
%
% If you use the hyperref package, please uncomment the following line
% to display URLs in blue roman font according to Springer's eBook style:
% \renewcommand\UrlFont{\color{blue}\rmfamily}
\newcommand{\TG}[1]{\noindent{\color{blue}[\textsc{From Tristan:} #1]}\xspace}
\newcommand{\Ali}[1]{\noindent{\color{green}[\textsc{From Ali:} #1]}\xspace}
\newcommand{\GK}[1]{\noindent{\color{purple}[\textsc{From Greg:} #1]}\xspace}
\newcommand{\YC}[1]{\noindent{\color{orange}[\textsc{From Yohan:} #1]}\xspace}
\newcommand{\anonymous}[1]{***\xspace}


\lstdefinestyle{customC}{
  belowcaptionskip=1\baselineskip,
  breaklines=true,
  xleftmargin=\parindent,
  language=C,
  showstringspaces=false,
  basicstyle=\scriptsize\ttfamily,
  keywordstyle=\bfseries\color[rgb]{0.580, 0.000, 0.827},
  %{purple!40!lightgray},
  commentstyle=\textit{\color{green!40!black}},
  identifierstyle=\bfseries\color{cyan!75!black},
  stringstyle=\color{orange},
  deletekeywords={double,float},
  classoffset=1, % starting new class
  otherkeywords={double,float},
  morekeywords={double,float},
  keywordstyle=\bfseries\color{green!55!black},
  classoffset=0
}

\begin{document}

\title{Comparing tool variability and numerical variability in fMRI analyses}

\author{Ali Salari$^1$, Co-authors$^2$, Tristan Glatard$^1$}

\authorrunning{Salari et al.}
% First names are abbreviated in the running head.
% If there are more than two authors, 'et al.' is used.

\institute{$^1$ Anonymous Organization1\\
  $^2$ Anonymous Organization2\\ **@******.***}

\institute{$^1$ Department of Computer Science and Software Engineering, Concordia University\\
  $^2$ Co-authors Organization\\  Montréal, QC, Canada}

\maketitle              % typeset the header of the contribution

\begin{abstract}
abstract\dots

%  \keywords{Computational reproducibility  \and Neuroimaging pipelines \and Monte-Carlo arithmetic.}
\end{abstract}


\section{Introduction}

\begin{description}
  \item[$\bullet$ ] Methodological choice can influence the final determining areas of brain activation. This opens so results flexibility~\cite{bowring2019exploring}. 

  \item[$\bullet$ ] We presented a framework that can investigate the numerical instability of the pipelines based on Monte-Carlo arithmetic
                    by creating a Fuzzy environment, so that instrument mathematical functions implemented in mathematical system libraries (libmath). 

  \item[$\bullet$ ] In this paper, we aim to answer the questions 1) how the fMRI analyses are numerically stable?
                    2) how the numerical variability is in comparison with the tool variability?
\end{description} 


\section{Methodology}

\subsection{libmath Fuzzy System}

\begin{description}
  \item[$\bullet$ ] libmath Fuzzy allows you to study the numerical stability of tools and pipelines. 
  \item[$\bullet$ ] It is based on MCA perturbations so that applies slight noise on floating-point operations.
  \item[$\bullet$ ] It does not need source code modification or recompilation, and simply works using the Linux LD\_PRELOAD environment variable.
  \item[$\bullet$ ] The library call interposition technique enables us to assess the tools that are dynamically linked to the mathematical library.
\end{description} 


\subsection{fMRI analyses \& Dataset}

\begin{description}
  \item[$\bullet$ ] We replicate the fMRI analyses used in~\cite{bowring2019exploring} using the three most popular packages 
                    for fMRI data processing including FSL, AFNI, and SPM. 
                    
  \item[$\bullet$ ] The functional fMRI studies have been chosen for reanalysis with the publicly available data repositories.
                    The used dataset is available at https://openneuro.org/datasets/ds000001.

  \item[$\bullet$ ] For the ds000001 study, 16 healthy adult subjects participated in a balloon analogue risk task over three scanning sessions.
                    This analysis consists of several common processes. A full description of each analysis is included in~\cite{bowring2019exploring}.

  \item[$\bullet$ ] In the original study, a number of preprocessing steps such as motion correction, segmentation, brain extraction, and registration 
                    were applied in all of the analyses to ensure that results from each software package could be compared objectively.
  
\end{description}  


\subsection{Data processing}

\begin{description}
  \item[$\bullet$ Containerization] building Docker images for three of the most popular software packages in neuroimaging including FSL, AFNI, and SPM.
                  We ensured that the software versions and all the requisites for running analyses used in all experiments 
                  are identical to the original study.  

  \item[$\bullet$ Between tool variability] check and confirm that obtained results are perfectly replicated comparing to the original study. 

  \item[$\bullet$ Numerical variability] running 3 MCA samples in each condition using the Fuzzy libmath environment. 
                  The conditions could be varying between tools, software versions, or the instrumented precisions. 
    \begin{description}
      \item[$\ast$ Varying precisions] a) only perturbing maximum precisions (OS-level), p=53 bits for double-precision and p=24 bits for single-precision.
                                       This would help to compare uncertainty between tool variability and numerical instability at the OS-level. 
                                       b) perturbing precisions from p=53 bits to p=1 bits by steps of 2 iteratively. 
                                       This would help to find the most similar uncertainty distribution between tool variability and numerical variability.
    \end{description}
    
  \item[$\bullet$ Image types] the fMRI result images that we are investigating are thresholded and unthresholded of the group level activation maps.

  \item[$\bullet$ Comparison metrics] comparing variabilities statistically by computing correlations between T-statistic values, Dice coefficient, 
                  and the number of significant digits.    

  \item[$\bullet$ Configuration table] showing the detail of the configurations used for replicating the fMRI results.
\end{description}


\section{Results}

\begin{description}
  \item[$\bullet$ Figure 1: Showing results visually] make an overlay of 3 samples of un/thresholded activations maps for each condition.

  \item[$\bullet$ Table 1: Summary of statistics] mean differences and correlations between T-statistic values for each pair of results in each condition.

  \item[$\bullet$ Figure 2: Bland-Altman] comparing unthresholded group level maps, 2D histogram, computed each pair of tools in both conditions.
                  Measures similarity between statistic values.
                  This plots difference between the statistic values (y-axis) and the mean statistic value (x-axis) for all voxels.

  \item[$\bullet$ Figure 3: Dice Coefficient] comparing the thresholded group level maps computed each pair of tools in both conditions.
                  Measures the overlap of voxels which assess the spatial similarity between activated maps. 

  \item[$\bullet$ Figure 4: Significant digits] comparing the distribution of SD across 3 samples (unthresholded images) for each condition. 
                  Find and show results at the precision that creates the most similar numerical uncertainty with between tool variability.  
\end{description}

% EC is a means to assess whether only superficial scaling differences (differences by a scale factor over all voxels) 
% are responsible for disparities between pair of images.



\section{Conclusion \& Discussion}
conclusions\dots
%
% ---- Bibliography ----
%
% BibTeX users should specify bibliography style 'splncs04'.
% References will then be sorted and formatted in the correct style.
%
\bibliographystyle{splncs04}
\bibliography{biblio}

\end{document}
